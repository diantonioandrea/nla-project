\subsection{Test Configuration}

\begin{frame}
    \frametitle{Test Configuration}

    The algorithm is tested using the following matrices from \href{https://sparse.tamu.edu}{SuiteSparse}:

    \begin{itemize}
        \item \textbf{\href{https://sparse.tamu.edu/CEMW/t2em}{t2em}}: 
        \begin{itemize}
            \item Non-symmetric square matrix.
            \item $n \approx \num{9.2e5}$, approximately $\num{4.5e6}$ non-zero entries.
            \item Origin: Electromagnetics problem.
        \end{itemize}
        \item \textbf{\href{https://sparse.tamu.edu/VLSI/stokes}{stokes}}: 
        \begin{itemize}
            \item Non-symmetric, ill-conditioned square matrix.
            \item $n \approx \num{1.1e7}$, approximately $\num{3.5e8}$ non-zero entries.
            \item Origin: Semiconductor process problem.
        \end{itemize}
    \end{itemize}

\end{frame}

\begin{frame}
    \frametitle{Right-Hand Side Vector and Testing Environment}

    \textbf{Right-hand side vector}
    \begin{itemize}
        \item $\Vector{b} \in \RealVectors{n}$ is computed as $\Vector{b} = \Matrix{A} \Vector{g}$.
        \item $\Vector{g} \in \RealVectors{n}$ is a random vector sampled from the standard normal distribution.
    \end{itemize}

    \textbf{Testing environment}
    \begin{itemize}
        \item Processor: AMD Ryzen 7 3700X.
        \item Memory: 64 GB.
    \end{itemize}

\end{frame}

\subsection{Results}

\begin{frame}
    \frametitle{\textbf{t2em} Matrix}

    % The \textit{k-sGMRES} algorithm demonstrates similar behavior to \lstinline{MATLAB}'s \textit{GMRES} in terms of relative residuals, while significantly outperforming it in computational time.

    \begin{figure}[!ht]
        \begin{subfigure}[b]{0.45\textwidth}
            \begin{tikzpicture}[scale=.6]
                \begin{loglogaxis}[
                    xlabel={Iterations $\left( d \right)$},
                    xtick={100, 250, 500, 1000, 2500, 5000},
                    xticklabels={100, 250, 500, 1000, 2500, 5000},
                    ylabel={Residual $\left( \Norm{\Matrix{A} \Vector{x} - \Vector{b}}_2 / \Norm{\Vector{b}}_2 \right)$},
                    grid=both,
                    legend pos=south west
                ]
                \addplot[color=\accentcolor, mark=*] coordinates {(100,6.786e-06) (250,4.270e-07) (500,4.052e-08) (1000,2.321e-09) (2500,1.266e-12) (5000,4.508e-15)};
                \addplot[color=black, mark=*] coordinates {(100,4.50e-06) (250,2.9e-07) (500,2.8e-08) (1000,1.6e-09) (2500,9e-13) (5000,7.8e-15)};
                \legend{\textit{k-sGMRES} (\lstinline{C++23}),\textit{GMRES} (\lstinline{MATLAB})}
                \end{loglogaxis}
            \end{tikzpicture}
        \end{subfigure}
        \hfill
        \begin{subfigure}[b]{0.45\textwidth}
            \begin{tikzpicture}[scale=.6]
                \begin{loglogaxis}[
                    xlabel={Iterations $\left( d \right)$},
                    xtick={100, 250, 500, 1000, 2500, 5000},
                    xticklabels={100, 250, 500, 1000, 2500, 5000},
                    ylabel={Time $\left[ \unit{ms} \right]$},
                    grid=both,
                    legend pos=south east
                ]
                \addplot[color=\accentcolor, mark=*] coordinates {(100,3083) (250,6753) (500,13550) (1000,28192) (2500,102517) (5000,285081)};
                \addplot[color=black, mark=*] coordinates {(100,9937) (250,54045) (500,209483) (1000,792628) (2500,4887066) (5000,10790146)};
                \legend{\textit{k-sGMRES} (\lstinline{C++23}),\textit{GMRES} (\lstinline{MATLAB})}
                \end{loglogaxis}
            \end{tikzpicture}
        \end{subfigure}
        \caption{Comparison of the performance of the \textit{k-sGMRES} algorithm with $k = 4$ against \lstinline{MATLAB}'s \textit{GMRES} algorithm for the \textbf{t2em} matrix, in terms of relative residual (left) and computation time (right).}
        \label{fig:t2em}
    \end{figure}

\end{frame}

\begin{frame}[fragile]
    \frametitle{\textbf{stokes} Matrix}

    % The \textit{k-sGMRES} algorithm demonstrates its ability to handle larger matrices, such as the \textbf{stokes} matrix. Preconditioning techniques can further enhance convergence.

    \begin{figure}[!ht]
        \begin{subfigure}[b]{0.45\textwidth}
            \begin{tikzpicture}[scale=.6]
                \begin{loglogaxis}[
                    xlabel={Iterations $\left( d \right)$},
                    xtick={250, 1000, 5000, 20000},
                    xticklabels={250, 1000, 5000, 20000},
                    ylabel={Residual $\left( \Norm{\Matrix{A} \Vector{x} - \Vector{b}}_2 / \Norm{\Vector{b}}_2 \right)$},
                    grid=both,
                    legend pos=north east
                ]
                \addplot[color=\accentcolor, mark=*] coordinates {(100,1.037e-03) (250,3.628e-04) (500,1.236e-04) (1000,4.016e-05) (2500,1.131e-05) (5000,2.377e-06) (10000,1.177e-06) (20000,5.020e-07)};
                \legend{\textit{k-sGMRES} (\lstinline{C++23})}
                \end{loglogaxis}
            \end{tikzpicture}
        \end{subfigure}
        \hfill
        \begin{subfigure}[b]{0.45\textwidth}
            \begin{tikzpicture}[scale=.6]
                \begin{loglogaxis}[
                        xlabel={Iterations $\left( d \right)$},
                        xtick={250, 1000, 5000, 20000},
                        xticklabels={250, 1000, 5000, 20000},
                        ylabel={Time $\left[ \unit{ms} \right]$},
                        grid=both,
                        legend pos=north west
                    ]
                    \addplot[color=\accentcolor, mark=*] coordinates {(100,113070) (250,281376) (500,568571)(1000,1147874) (2500,3030201) (5000,6211285) (10000,13173049) (20000,31922165)};
                    \legend{\textit{k-sGMRES} (\lstinline{C++23})}
                \end{loglogaxis}
            \end{tikzpicture}
        \end{subfigure}
        \caption{Evaluation of the performance of the \textit{k-sGMRES} algorithm with $k = 4$ for the \textbf{stokes} matrix, in terms of relative residual (left) and computation time (right).}
        \label{fig:stokes}
    \end{figure}
    
\end{frame}

% \begin{frame}{Comparison with \lstinline{MATLAB}'s \textit{GMRES}}
%     \begin{itemize}
%         \item \textbf{Performance Overview}
%         \begin{itemize}
%             \item The \textit{k-sGMRES} algorithm achieves similar relative residuals as \lstinline{MATLAB}'s \textit{GMRES}.
%             \item It significantly outperforms \lstinline{MATLAB}'s \textit{GMRES} in terms of computational time.
%         \end{itemize}

%         \item \textbf{Handling Larger Matrices}
%         \begin{itemize}
%             \item The \textit{k-sGMRES} algorithm efficiently handles large matrices, such as the \textbf{stokes} matrix.
%             \item Preconditioning techniques can further enhance its convergence.
%         \end{itemize}
        
%         \item \textbf{Limitations of \lstinline{MATLAB}'s \textit{GMRES}}
%         \begin{itemize}
%             \item For the \textbf{stokes} matrix, \lstinline{MATLAB}'s \textit{GMRES} fails to converge within a reasonable time frame.
%             \item This highlights the superior efficiency of the \textit{k-sGMRES} algorithm for large-scale problems.
%         \end{itemize}
%     \end{itemize}
% \end{frame}